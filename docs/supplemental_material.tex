\documentclass[11pt]{article}

    \usepackage[breakable]{tcolorbox}
    \usepackage{parskip} % Stop auto-indenting (to mimic markdown behaviour)
    

    % Basic figure setup, for now with no caption control since it's done
    % automatically by Pandoc (which extracts ![](path) syntax from Markdown).
    \usepackage{graphicx}
    % Maintain compatibility with old templates. Remove in nbconvert 6.0
    \let\Oldincludegraphics\includegraphics
    % Ensure that by default, figures have no caption (until we provide a
    % proper Figure object with a Caption API and a way to capture that
    % in the conversion process - todo).
    \usepackage{caption}
    \DeclareCaptionFormat{nocaption}{}
    \captionsetup{format=nocaption,aboveskip=0pt,belowskip=0pt}

    \usepackage{float}
    \floatplacement{figure}{H} % forces figures to be placed at the correct location
    \usepackage{xcolor} % Allow colors to be defined
    \usepackage{enumerate} % Needed for markdown enumerations to work
    \usepackage{geometry} % Used to adjust the document margins
    \usepackage{amsmath} % Equations
    \usepackage{amssymb} % Equations
    \usepackage{textcomp} % defines textquotesingle
    % Hack from http://tex.stackexchange.com/a/47451/13684:
    \AtBeginDocument{%
        \def\PYZsq{\textquotesingle}% Upright quotes in Pygmentized code
    }
    \usepackage{upquote} % Upright quotes for verbatim code
    \usepackage{eurosym} % defines \euro

    \usepackage{iftex}
    \ifPDFTeX
        \usepackage[T1]{fontenc}
        \IfFileExists{alphabeta.sty}{
              \usepackage{alphabeta}
          }{
              \usepackage[mathletters]{ucs}
              \usepackage[utf8x]{inputenc}
          }
    \else
        \usepackage{fontspec}
        \usepackage{unicode-math}
    \fi

    \usepackage{fancyvrb} % verbatim replacement that allows latex
    \usepackage{grffile} % extends the file name processing of package graphics
                         % to support a larger range
    \makeatletter % fix for old versions of grffile with XeLaTeX
    \@ifpackagelater{grffile}{2019/11/01}
    {
      % Do nothing on new versions
    }
    {
      \def\Gread@@xetex#1{%
        \IfFileExists{"\Gin@base".bb}%
        {\Gread@eps{\Gin@base.bb}}%
        {\Gread@@xetex@aux#1}%
      }
    }
    \makeatother
    \usepackage[Export]{adjustbox} % Used to constrain images to a maximum size
    \adjustboxset{max size={0.9\linewidth}{0.9\paperheight}}

    % The hyperref package gives us a pdf with properly built
    % internal navigation ('pdf bookmarks' for the table of contents,
    % internal cross-reference links, web links for URLs, etc.)
    \usepackage{hyperref}
    % The default LaTeX title has an obnoxious amount of whitespace. By default,
    % titling removes some of it. It also provides customization options.
    \usepackage{titling}
    \usepackage{longtable} % longtable support required by pandoc >1.10
    \usepackage{booktabs}  % table support for pandoc > 1.12.2
    \usepackage{array}     % table support for pandoc >= 2.11.3
    \usepackage{calc}      % table minipage width calculation for pandoc >= 2.11.1
    \usepackage[inline]{enumitem} % IRkernel/repr support (it uses the enumerate* environment)
    \usepackage[normalem]{ulem} % ulem is needed to support strikethroughs (\sout)
                                % normalem makes italics be italics, not underlines
    \usepackage{mathrsfs}
    

    
    % Colors for the hyperref package
    \definecolor{urlcolor}{rgb}{0,.145,.698}
    \definecolor{linkcolor}{rgb}{.71,0.21,0.01}
    \definecolor{citecolor}{rgb}{.12,.54,.11}

    % ANSI colors
    \definecolor{ansi-black}{HTML}{3E424D}
    \definecolor{ansi-black-intense}{HTML}{282C36}
    \definecolor{ansi-red}{HTML}{E75C58}
    \definecolor{ansi-red-intense}{HTML}{B22B31}
    \definecolor{ansi-green}{HTML}{00A250}
    \definecolor{ansi-green-intense}{HTML}{007427}
    \definecolor{ansi-yellow}{HTML}{DDB62B}
    \definecolor{ansi-yellow-intense}{HTML}{B27D12}
    \definecolor{ansi-blue}{HTML}{208FFB}
    \definecolor{ansi-blue-intense}{HTML}{0065CA}
    \definecolor{ansi-magenta}{HTML}{D160C4}
    \definecolor{ansi-magenta-intense}{HTML}{A03196}
    \definecolor{ansi-cyan}{HTML}{60C6C8}
    \definecolor{ansi-cyan-intense}{HTML}{258F8F}
    \definecolor{ansi-white}{HTML}{C5C1B4}
    \definecolor{ansi-white-intense}{HTML}{A1A6B2}
    \definecolor{ansi-default-inverse-fg}{HTML}{FFFFFF}
    \definecolor{ansi-default-inverse-bg}{HTML}{000000}

    % common color for the border for error outputs.
    \definecolor{outerrorbackground}{HTML}{FFDFDF}

    % commands and environments needed by pandoc snippets
    % extracted from the output of `pandoc -s`
    \providecommand{\tightlist}{%
      \setlength{\itemsep}{0pt}\setlength{\parskip}{0pt}}
    \DefineVerbatimEnvironment{Highlighting}{Verbatim}{commandchars=\\\{\}}
    % Add ',fontsize=\small' for more characters per line
    \newenvironment{Shaded}{}{}
    \newcommand{\KeywordTok}[1]{\textcolor[rgb]{0.00,0.44,0.13}{\textbf{{#1}}}}
    \newcommand{\DataTypeTok}[1]{\textcolor[rgb]{0.56,0.13,0.00}{{#1}}}
    \newcommand{\DecValTok}[1]{\textcolor[rgb]{0.25,0.63,0.44}{{#1}}}
    \newcommand{\BaseNTok}[1]{\textcolor[rgb]{0.25,0.63,0.44}{{#1}}}
    \newcommand{\FloatTok}[1]{\textcolor[rgb]{0.25,0.63,0.44}{{#1}}}
    \newcommand{\CharTok}[1]{\textcolor[rgb]{0.25,0.44,0.63}{{#1}}}
    \newcommand{\StringTok}[1]{\textcolor[rgb]{0.25,0.44,0.63}{{#1}}}
    \newcommand{\CommentTok}[1]{\textcolor[rgb]{0.38,0.63,0.69}{\textit{{#1}}}}
    \newcommand{\OtherTok}[1]{\textcolor[rgb]{0.00,0.44,0.13}{{#1}}}
    \newcommand{\AlertTok}[1]{\textcolor[rgb]{1.00,0.00,0.00}{\textbf{{#1}}}}
    \newcommand{\FunctionTok}[1]{\textcolor[rgb]{0.02,0.16,0.49}{{#1}}}
    \newcommand{\RegionMarkerTok}[1]{{#1}}
    \newcommand{\ErrorTok}[1]{\textcolor[rgb]{1.00,0.00,0.00}{\textbf{{#1}}}}
    \newcommand{\NormalTok}[1]{{#1}}

    % Additional commands for more recent versions of Pandoc
    \newcommand{\ConstantTok}[1]{\textcolor[rgb]{0.53,0.00,0.00}{{#1}}}
    \newcommand{\SpecialCharTok}[1]{\textcolor[rgb]{0.25,0.44,0.63}{{#1}}}
    \newcommand{\VerbatimStringTok}[1]{\textcolor[rgb]{0.25,0.44,0.63}{{#1}}}
    \newcommand{\SpecialStringTok}[1]{\textcolor[rgb]{0.73,0.40,0.53}{{#1}}}
    \newcommand{\ImportTok}[1]{{#1}}
    \newcommand{\DocumentationTok}[1]{\textcolor[rgb]{0.73,0.13,0.13}{\textit{{#1}}}}
    \newcommand{\AnnotationTok}[1]{\textcolor[rgb]{0.38,0.63,0.69}{\textbf{\textit{{#1}}}}}
    \newcommand{\CommentVarTok}[1]{\textcolor[rgb]{0.38,0.63,0.69}{\textbf{\textit{{#1}}}}}
    \newcommand{\VariableTok}[1]{\textcolor[rgb]{0.10,0.09,0.49}{{#1}}}
    \newcommand{\ControlFlowTok}[1]{\textcolor[rgb]{0.00,0.44,0.13}{\textbf{{#1}}}}
    \newcommand{\OperatorTok}[1]{\textcolor[rgb]{0.40,0.40,0.40}{{#1}}}
    \newcommand{\BuiltInTok}[1]{{#1}}
    \newcommand{\ExtensionTok}[1]{{#1}}
    \newcommand{\PreprocessorTok}[1]{\textcolor[rgb]{0.74,0.48,0.00}{{#1}}}
    \newcommand{\AttributeTok}[1]{\textcolor[rgb]{0.49,0.56,0.16}{{#1}}}
    \newcommand{\InformationTok}[1]{\textcolor[rgb]{0.38,0.63,0.69}{\textbf{\textit{{#1}}}}}
    \newcommand{\WarningTok}[1]{\textcolor[rgb]{0.38,0.63,0.69}{\textbf{\textit{{#1}}}}}


    % Define a nice break command that doesn't care if a line doesn't already
    % exist.
    \def\br{\hspace*{\fill} \\* }
    % Math Jax compatibility definitions
    \def\gt{>}
    \def\lt{<}
    \let\Oldtex\TeX
    \let\Oldlatex\LaTeX
    \renewcommand{\TeX}{\textrm{\Oldtex}}
    \renewcommand{\LaTeX}{\textrm{\Oldlatex}}
    % Document parameters
    % Document title
\title{An Oscillating Reaction Network With an Exact, Closed-Form Time-Domain Solution: Derivation of the Time Domain Solution}

\author{Joseph L. Hellerstein\\ eScience Institute\\ University of Washington}
    
    
    
    
    
% Pygments definitions
\makeatletter
\def\PY@reset{\let\PY@it=\relax \let\PY@bf=\relax%
    \let\PY@ul=\relax \let\PY@tc=\relax%
    \let\PY@bc=\relax \let\PY@ff=\relax}
\def\PY@tok#1{\csname PY@tok@#1\endcsname}
\def\PY@toks#1+{\ifx\relax#1\empty\else%
    \PY@tok{#1}\expandafter\PY@toks\fi}
\def\PY@do#1{\PY@bc{\PY@tc{\PY@ul{%
    \PY@it{\PY@bf{\PY@ff{#1}}}}}}}
\def\PY#1#2{\PY@reset\PY@toks#1+\relax+\PY@do{#2}}

\@namedef{PY@tok@w}{\def\PY@tc##1{\textcolor[rgb]{0.73,0.73,0.73}{##1}}}
\@namedef{PY@tok@c}{\let\PY@it=\textit\def\PY@tc##1{\textcolor[rgb]{0.24,0.48,0.48}{##1}}}
\@namedef{PY@tok@cp}{\def\PY@tc##1{\textcolor[rgb]{0.61,0.40,0.00}{##1}}}
\@namedef{PY@tok@k}{\let\PY@bf=\textbf\def\PY@tc##1{\textcolor[rgb]{0.00,0.50,0.00}{##1}}}
\@namedef{PY@tok@kp}{\def\PY@tc##1{\textcolor[rgb]{0.00,0.50,0.00}{##1}}}
\@namedef{PY@tok@kt}{\def\PY@tc##1{\textcolor[rgb]{0.69,0.00,0.25}{##1}}}
\@namedef{PY@tok@o}{\def\PY@tc##1{\textcolor[rgb]{0.40,0.40,0.40}{##1}}}
\@namedef{PY@tok@ow}{\let\PY@bf=\textbf\def\PY@tc##1{\textcolor[rgb]{0.67,0.13,1.00}{##1}}}
\@namedef{PY@tok@nb}{\def\PY@tc##1{\textcolor[rgb]{0.00,0.50,0.00}{##1}}}
\@namedef{PY@tok@nf}{\def\PY@tc##1{\textcolor[rgb]{0.00,0.00,1.00}{##1}}}
\@namedef{PY@tok@nc}{\let\PY@bf=\textbf\def\PY@tc##1{\textcolor[rgb]{0.00,0.00,1.00}{##1}}}
\@namedef{PY@tok@nn}{\let\PY@bf=\textbf\def\PY@tc##1{\textcolor[rgb]{0.00,0.00,1.00}{##1}}}
\@namedef{PY@tok@ne}{\let\PY@bf=\textbf\def\PY@tc##1{\textcolor[rgb]{0.80,0.25,0.22}{##1}}}
\@namedef{PY@tok@nv}{\def\PY@tc##1{\textcolor[rgb]{0.10,0.09,0.49}{##1}}}
\@namedef{PY@tok@no}{\def\PY@tc##1{\textcolor[rgb]{0.53,0.00,0.00}{##1}}}
\@namedef{PY@tok@nl}{\def\PY@tc##1{\textcolor[rgb]{0.46,0.46,0.00}{##1}}}
\@namedef{PY@tok@ni}{\let\PY@bf=\textbf\def\PY@tc##1{\textcolor[rgb]{0.44,0.44,0.44}{##1}}}
\@namedef{PY@tok@na}{\def\PY@tc##1{\textcolor[rgb]{0.41,0.47,0.13}{##1}}}
\@namedef{PY@tok@nt}{\let\PY@bf=\textbf\def\PY@tc##1{\textcolor[rgb]{0.00,0.50,0.00}{##1}}}
\@namedef{PY@tok@nd}{\def\PY@tc##1{\textcolor[rgb]{0.67,0.13,1.00}{##1}}}
\@namedef{PY@tok@s}{\def\PY@tc##1{\textcolor[rgb]{0.73,0.13,0.13}{##1}}}
\@namedef{PY@tok@sd}{\let\PY@it=\textit\def\PY@tc##1{\textcolor[rgb]{0.73,0.13,0.13}{##1}}}
\@namedef{PY@tok@si}{\let\PY@bf=\textbf\def\PY@tc##1{\textcolor[rgb]{0.64,0.35,0.47}{##1}}}
\@namedef{PY@tok@se}{\let\PY@bf=\textbf\def\PY@tc##1{\textcolor[rgb]{0.67,0.36,0.12}{##1}}}
\@namedef{PY@tok@sr}{\def\PY@tc##1{\textcolor[rgb]{0.64,0.35,0.47}{##1}}}
\@namedef{PY@tok@ss}{\def\PY@tc##1{\textcolor[rgb]{0.10,0.09,0.49}{##1}}}
\@namedef{PY@tok@sx}{\def\PY@tc##1{\textcolor[rgb]{0.00,0.50,0.00}{##1}}}
\@namedef{PY@tok@m}{\def\PY@tc##1{\textcolor[rgb]{0.40,0.40,0.40}{##1}}}
\@namedef{PY@tok@gh}{\let\PY@bf=\textbf\def\PY@tc##1{\textcolor[rgb]{0.00,0.00,0.50}{##1}}}
\@namedef{PY@tok@gu}{\let\PY@bf=\textbf\def\PY@tc##1{\textcolor[rgb]{0.50,0.00,0.50}{##1}}}
\@namedef{PY@tok@gd}{\def\PY@tc##1{\textcolor[rgb]{0.63,0.00,0.00}{##1}}}
\@namedef{PY@tok@gi}{\def\PY@tc##1{\textcolor[rgb]{0.00,0.52,0.00}{##1}}}
\@namedef{PY@tok@gr}{\def\PY@tc##1{\textcolor[rgb]{0.89,0.00,0.00}{##1}}}
\@namedef{PY@tok@ge}{\let\PY@it=\textit}
\@namedef{PY@tok@gs}{\let\PY@bf=\textbf}
\@namedef{PY@tok@gp}{\let\PY@bf=\textbf\def\PY@tc##1{\textcolor[rgb]{0.00,0.00,0.50}{##1}}}
\@namedef{PY@tok@go}{\def\PY@tc##1{\textcolor[rgb]{0.44,0.44,0.44}{##1}}}
\@namedef{PY@tok@gt}{\def\PY@tc##1{\textcolor[rgb]{0.00,0.27,0.87}{##1}}}
\@namedef{PY@tok@err}{\def\PY@bc##1{{\setlength{\fboxsep}{\string -\fboxrule}\fcolorbox[rgb]{1.00,0.00,0.00}{1,1,1}{\strut ##1}}}}
\@namedef{PY@tok@kc}{\let\PY@bf=\textbf\def\PY@tc##1{\textcolor[rgb]{0.00,0.50,0.00}{##1}}}
\@namedef{PY@tok@kd}{\let\PY@bf=\textbf\def\PY@tc##1{\textcolor[rgb]{0.00,0.50,0.00}{##1}}}
\@namedef{PY@tok@kn}{\let\PY@bf=\textbf\def\PY@tc##1{\textcolor[rgb]{0.00,0.50,0.00}{##1}}}
\@namedef{PY@tok@kr}{\let\PY@bf=\textbf\def\PY@tc##1{\textcolor[rgb]{0.00,0.50,0.00}{##1}}}
\@namedef{PY@tok@bp}{\def\PY@tc##1{\textcolor[rgb]{0.00,0.50,0.00}{##1}}}
\@namedef{PY@tok@fm}{\def\PY@tc##1{\textcolor[rgb]{0.00,0.00,1.00}{##1}}}
\@namedef{PY@tok@vc}{\def\PY@tc##1{\textcolor[rgb]{0.10,0.09,0.49}{##1}}}
\@namedef{PY@tok@vg}{\def\PY@tc##1{\textcolor[rgb]{0.10,0.09,0.49}{##1}}}
\@namedef{PY@tok@vi}{\def\PY@tc##1{\textcolor[rgb]{0.10,0.09,0.49}{##1}}}
\@namedef{PY@tok@vm}{\def\PY@tc##1{\textcolor[rgb]{0.10,0.09,0.49}{##1}}}
\@namedef{PY@tok@sa}{\def\PY@tc##1{\textcolor[rgb]{0.73,0.13,0.13}{##1}}}
\@namedef{PY@tok@sb}{\def\PY@tc##1{\textcolor[rgb]{0.73,0.13,0.13}{##1}}}
\@namedef{PY@tok@sc}{\def\PY@tc##1{\textcolor[rgb]{0.73,0.13,0.13}{##1}}}
\@namedef{PY@tok@dl}{\def\PY@tc##1{\textcolor[rgb]{0.73,0.13,0.13}{##1}}}
\@namedef{PY@tok@s2}{\def\PY@tc##1{\textcolor[rgb]{0.73,0.13,0.13}{##1}}}
\@namedef{PY@tok@sh}{\def\PY@tc##1{\textcolor[rgb]{0.73,0.13,0.13}{##1}}}
\@namedef{PY@tok@s1}{\def\PY@tc##1{\textcolor[rgb]{0.73,0.13,0.13}{##1}}}
\@namedef{PY@tok@mb}{\def\PY@tc##1{\textcolor[rgb]{0.40,0.40,0.40}{##1}}}
\@namedef{PY@tok@mf}{\def\PY@tc##1{\textcolor[rgb]{0.40,0.40,0.40}{##1}}}
\@namedef{PY@tok@mh}{\def\PY@tc##1{\textcolor[rgb]{0.40,0.40,0.40}{##1}}}
\@namedef{PY@tok@mi}{\def\PY@tc##1{\textcolor[rgb]{0.40,0.40,0.40}{##1}}}
\@namedef{PY@tok@il}{\def\PY@tc##1{\textcolor[rgb]{0.40,0.40,0.40}{##1}}}
\@namedef{PY@tok@mo}{\def\PY@tc##1{\textcolor[rgb]{0.40,0.40,0.40}{##1}}}
\@namedef{PY@tok@ch}{\let\PY@it=\textit\def\PY@tc##1{\textcolor[rgb]{0.24,0.48,0.48}{##1}}}
\@namedef{PY@tok@cm}{\let\PY@it=\textit\def\PY@tc##1{\textcolor[rgb]{0.24,0.48,0.48}{##1}}}
\@namedef{PY@tok@cpf}{\let\PY@it=\textit\def\PY@tc##1{\textcolor[rgb]{0.24,0.48,0.48}{##1}}}
\@namedef{PY@tok@c1}{\let\PY@it=\textit\def\PY@tc##1{\textcolor[rgb]{0.24,0.48,0.48}{##1}}}
\@namedef{PY@tok@cs}{\let\PY@it=\textit\def\PY@tc##1{\textcolor[rgb]{0.24,0.48,0.48}{##1}}}

\def\PYZbs{\char`\\}
\def\PYZus{\char`\_}
\def\PYZob{\char`\{}
\def\PYZcb{\char`\}}
\def\PYZca{\char`\^}
\def\PYZam{\char`\&}
\def\PYZlt{\char`\<}
\def\PYZgt{\char`\>}
\def\PYZsh{\char`\#}
\def\PYZpc{\char`\%}
\def\PYZdl{\char`\$}
\def\PYZhy{\char`\-}
\def\PYZsq{\char`\'}
\def\PYZdq{\char`\"}
\def\PYZti{\char`\~}
% for compatibility with earlier versions
\def\PYZat{@}
\def\PYZlb{[}
\def\PYZrb{]}
\makeatother


    % For linebreaks inside Verbatim environment from package fancyvrb.
    \makeatletter
        \newbox\Wrappedcontinuationbox
        \newbox\Wrappedvisiblespacebox
        \newcommand*\Wrappedvisiblespace {\textcolor{red}{\textvisiblespace}}
        \newcommand*\Wrappedcontinuationsymbol {\textcolor{red}{\llap{\tiny$\m@th\hookrightarrow$}}}
        \newcommand*\Wrappedcontinuationindent {3ex }
        \newcommand*\Wrappedafterbreak {\kern\Wrappedcontinuationindent\copy\Wrappedcontinuationbox}
        % Take advantage of the already applied Pygments mark-up to insert
        % potential linebreaks for TeX processing.
        %        {, <, #, %, $, ' and ": go to next line.
        %        _, }, ^, &, >, - and ~: stay at end of broken line.
        % Use of \textquotesingle for straight quote.
        \newcommand*\Wrappedbreaksatspecials {%
            \def\PYGZus{\discretionary{\char`\_}{\Wrappedafterbreak}{\char`\_}}%
            \def\PYGZob{\discretionary{}{\Wrappedafterbreak\char`\{}{\char`\{}}%
            \def\PYGZcb{\discretionary{\char`\}}{\Wrappedafterbreak}{\char`\}}}%
            \def\PYGZca{\discretionary{\char`\^}{\Wrappedafterbreak}{\char`\^}}%
            \def\PYGZam{\discretionary{\char`\&}{\Wrappedafterbreak}{\char`\&}}%
            \def\PYGZlt{\discretionary{}{\Wrappedafterbreak\char`\<}{\char`\<}}%
            \def\PYGZgt{\discretionary{\char`\>}{\Wrappedafterbreak}{\char`\>}}%
            \def\PYGZsh{\discretionary{}{\Wrappedafterbreak\char`\#}{\char`\#}}%
            \def\PYGZpc{\discretionary{}{\Wrappedafterbreak\char`\%}{\char`\%}}%
            \def\PYGZdl{\discretionary{}{\Wrappedafterbreak\char`\$}{\char`\$}}%
            \def\PYGZhy{\discretionary{\char`\-}{\Wrappedafterbreak}{\char`\-}}%
            \def\PYGZsq{\discretionary{}{\Wrappedafterbreak\textquotesingle}{\textquotesingle}}%
            \def\PYGZdq{\discretionary{}{\Wrappedafterbreak\char`\"}{\char`\"}}%
            \def\PYGZti{\discretionary{\char`\~}{\Wrappedafterbreak}{\char`\~}}%
        }
        % Some characters . , ; ? ! / are not pygmentized.
        % This macro makes them "active" and they will insert potential linebreaks
        \newcommand*\Wrappedbreaksatpunct {%
            \lccode`\~`\.\lowercase{\def~}{\discretionary{\hbox{\char`\.}}{\Wrappedafterbreak}{\hbox{\char`\.}}}%
            \lccode`\~`\,\lowercase{\def~}{\discretionary{\hbox{\char`\,}}{\Wrappedafterbreak}{\hbox{\char`\,}}}%
            \lccode`\~`\;\lowercase{\def~}{\discretionary{\hbox{\char`\;}}{\Wrappedafterbreak}{\hbox{\char`\;}}}%
            \lccode`\~`\:\lowercase{\def~}{\discretionary{\hbox{\char`\:}}{\Wrappedafterbreak}{\hbox{\char`\:}}}%
            \lccode`\~`\?\lowercase{\def~}{\discretionary{\hbox{\char`\?}}{\Wrappedafterbreak}{\hbox{\char`\?}}}%
            \lccode`\~`\!\lowercase{\def~}{\discretionary{\hbox{\char`\!}}{\Wrappedafterbreak}{\hbox{\char`\!}}}%
            \lccode`\~`\/\lowercase{\def~}{\discretionary{\hbox{\char`\/}}{\Wrappedafterbreak}{\hbox{\char`\/}}}%
            \catcode`\.\active
            \catcode`\,\active
            \catcode`\;\active
            \catcode`\:\active
            \catcode`\?\active
            \catcode`\!\active
            \catcode`\/\active
            \lccode`\~`\~
        }
    \makeatother

    \let\OriginalVerbatim=\Verbatim
    \makeatletter
    \renewcommand{\Verbatim}[1][1]{%
        %\parskip\z@skip
        \sbox\Wrappedcontinuationbox {\Wrappedcontinuationsymbol}%
        \sbox\Wrappedvisiblespacebox {\FV@SetupFont\Wrappedvisiblespace}%
        \def\FancyVerbFormatLine ##1{\hsize\linewidth
            \vtop{\raggedright\hyphenpenalty\z@\exhyphenpenalty\z@
                \doublehyphendemerits\z@\finalhyphendemerits\z@
                \strut ##1\strut}%
        }%
        % If the linebreak is at a space, the latter will be displayed as visible
        % space at end of first line, and a continuation symbol starts next line.
        % Stretch/shrink are however usually zero for typewriter font.
        \def\FV@Space {%
            \nobreak\hskip\z@ plus\fontdimen3\font minus\fontdimen4\font
            \discretionary{\copy\Wrappedvisiblespacebox}{\Wrappedafterbreak}
            {\kern\fontdimen2\font}%
        }%

        % Allow breaks at special characters using \PYG... macros.
        \Wrappedbreaksatspecials
        % Breaks at punctuation characters . , ; ? ! and / need catcode=\active
        \OriginalVerbatim[#1,codes*=\Wrappedbreaksatpunct]%
    }
    \makeatother

    % Exact colors from NB
    \definecolor{incolor}{HTML}{303F9F}
    \definecolor{outcolor}{HTML}{D84315}
    \definecolor{cellborder}{HTML}{CFCFCF}
    \definecolor{cellbackground}{HTML}{F7F7F7}

    % prompt
    \makeatletter
    \newcommand{\boxspacing}{\kern\kvtcb@left@rule\kern\kvtcb@boxsep}
    \makeatother
    \newcommand{\prompt}[4]{
        {\ttfamily\llap{{\color{#2}[#3]:\hspace{3pt}#4}}\vspace{-\baselineskip}}
    }
    

    
    % Prevent overflowing lines due to hard-to-break entities
    \sloppy
    % Setup hyperref package
    \hypersetup{
      breaklinks=true,  % so long urls are correctly broken across lines
      colorlinks=true,
      urlcolor=urlcolor,
      linkcolor=linkcolor,
      citecolor=citecolor,
      }
    % Slightly bigger margins than the latex defaults
    
    \geometry{verbose,tmargin=1in,bmargin=1in,lmargin=1in,rmargin=1in}
    
    

\begin{document}
    
    \maketitle
    
%%%%%%%%%%%%%%%%%%%%%%%%%%%%%%%%%%%
\section{Preliminaries}

    This write-up provides details on the derivation of the time domain solution for the reaction network described in the main text.

    The reaction network has 8 parameters. There are 6 kinetic constants,
\(k_i,~ i\in \{1, 2, 3, 4, 5, 6\}\). And there are two initial
concentrations, \(x_n (0),~ n \in \{1, 2 \}\), where \(x_n(0)\) is the
initial concentration for \(S_n\).

Because the reaction network can be described as a 
system of linear system (by construction), we
know that oscillations are sinusoids.
Let \(x_n (t)\) be the concentration
of species \(S_n\) at time \(t\). Then, an oscillating solution has the
form \begin{equation*}
x_n(t) = \alpha_n sin(\theta_n t + \phi_n) + \omega_n,
\end{equation*} where \(\alpha_n\) is the amplitude of the sinusoid for
\(S_n\), \(\theta_n\) is its frequency, \(\phi_n\) is its phase, and
\(\omega_n\) is the DC offset (the mean value of the sinusoid over
time).
It turns out that both species have the same frequency, as we show shortly.
So, $\theta_n = \theta$, and there are just 7 parameters.
We refer to \(\alpha_n, \theta, \phi_n, \omega_n\) as the
\textbf{oscillation characteristics (OC)} of the reaction network.

We use the following notation: 
\begin{itemize}
\item \({\bf A}\) - Jacobian matrix
\item \(\alpha_n\) - amplitude of oscillation for species \(n\)
\item \(\Delta\) -
\(det {\bf A})\)
\item \(i\) - indexes constants
\item \(k_i\), \(k_d\) -
positive constant 
\item \(K\) - number of constants
\item \(\lambda\) -
eigenvalue
\item \(n\) - indexes species
\item \(N\) - number of species
\item \(\omega_n\) - offset of species \(n\) 
\item \(\phi_n\) - phase in radians
\item \(t\) - time 
\item \(\tau\) - \(tr({\bf A})\) 
\item \(\theta\) - frequency in
radians 
\item \({\bf u}\) - forced input (kinetic constants for zeroth order
rates)  \({\bf x}\) (\(N \times 1\)) is the state vector
\item \(\dot{\bf x} (t)\) - derivative w.r.t. time of \({\bf x}\)
\item \(x_n\)
(t) - time varying concentration of species \(n\)
\end{itemize}

    Since we have constructed the reaction network so that the dynamics can
be described as a system of linear ODEs, it can be described using the
vector differential equation: \begin{equation}
\dot{\bf x} = {\bf A} {\bf x} + {\bf u}\label{eq:linear}
\end{equation} where
\({\bf A} = \begin{pmatrix} a_{11} & a_{12} \\ a_{21} & a_{22} \\ \end{pmatrix}\)

%%%%%%%%%%%%%%%%%%%%%%%%%%%%%%%%%%%
\section{Eigenvalues}
Here, we calculate the eigenvalues for the reaction network described
in the text.

If there is an oscillating solution for this system, then the
eigenvalues of \({\bf A}\) must be pure imaginary. Since this is a two
state system, this means that if \(\theta i\) is an eigenvalue, then
\(-\theta i\) must also be an eigenvalue. This means that
\(\theta_1 = \theta = \theta_2\), which justifies our previous claim.

Next we develop the conditions for ${\bf A}$ to have a pure imaginary
eigenvalues. The determinant of ${\bf A}$ is
$det({\bf A}) = a_{11} a_{22} - a_{12} a_{21} = \Delta.$
The trace of
${\bf A}$ is $tr({\bf A}) = a_{11} + a_{22} = \tau.$
So, the eigenvalues $\lambda$ are in
$\frac{1}{2} \left( - \tau \pm \sqrt{\tau^2 - 4 \Delta} \right). $
We get pure imaginary eigenvalues 
if the following constraints hold:
\begin{itemize}
\item $\tau = 0$ 
\item $\Delta > 0.$
\end{itemize}

Now, we analyze ${\bf A}$ for the reaction network.
\begin{equation}
{\bf A} =
\begin{pmatrix}
k_3 - k_1 & k_2 \\
k_1 - k_5 & -k_2 \\
\end{pmatrix}
\end{equation}

and 
\begin{equation}
{\bf u} = 
\begin{pmatrix} - k_4 \\ k_6 \\ \end{pmatrix}
\end{equation}
So the trace and determinant are:
\begin{eqnarray}
\tau & = & k_3 -k_1 - k_2 \\
\Delta & = & (k_3 - k_1)(-k_2) - k_2 (k_1 - k_5) \\
& = & k_2 (k_5 - k_3) \\
\end{eqnarray}

To obtain purely imaginary solutions, we require that \(\tau =0\) and
\(\Delta > 0\). The former implies that \(k_3 = k_1 + k_2\). The latter
implies that that \(k_5 > k_3\). We define \(k_d = k_5 - k_3 > 0\)
Applying the foregoing to the \({\bf A}\) matrix, we first note that
\begin{align*}
k_1 - k_5 & = & k_1 - k_3 -k_d \\
& = & k_3 - k_2 - k_3 - k_d \\
& = & -k_2 - k_d \\
\end{align*}
Applying the constraints,
we see that
\begin{equation}
{\bf A} = 
\begin{pmatrix}
k_2 & k_2 \\
-k_2 - k_d & -k_2 \\
\end{pmatrix}
\end{equation}
Observe that $\(\Delta = k_2 k_d$.
As a result
\(\theta = \pm \sqrt{\Delta} = \pm \sqrt{k_2 k_d}\). Hereafter, we drop
the \(\pm\).

%%%%%%%%%%%%%%%%%%%%%%%%
\section{Eigenvectors and Fundamental Matrix}

We find the eigenvectors of \({\bf A}\) as an intermediate step to
finding the time domain solution.

First, observe that that since \(k_d > 0\), \({\bf A}\) is nonsingular,
and so we can calculate eigenvectors directly.

\begin{equation}
    {\bf w}_1 =
    \begin{pmatrix}
- \frac{k_{2}}{k_{2} + k_{d}}
+ \frac{i \theta}{k_{2} + k_{d}} 
\\ 1
\end{pmatrix}
\end{equation}
for the eigenvalue $\lambda_1 = - \sqrt{k_d k_2}$.

\begin{equation}
    {\bf w}_2 =
    \begin{pmatrix}
- \frac{k_{2}}{k_{2} + k_{d}}
- \frac{i \theta}{k_{2} + k_{d}} 
\\ 1
\end{pmatrix}
\end{equation}
for the eigenvalue $\lambda_1 = \sqrt{k_d k_2}$.


Note that ${\bf A}$ is real valued and the eigenvectors are complex
conjuates. For example,
\begin{equation}
{\bf w}_ 1 = 
\begin{pmatrix}
- \frac{k_{2}}{k_{2} + k_{d}}
\\ 1
\end{pmatrix}
+
i
\begin{pmatrix}
 \frac{\theta}{k_{2} + k_{d}} 
\\ 0
\end{pmatrix}
\end{equation}
We have a similar decomposition for ${\bf w}_2$.
Now consider ${\bf w}_n e^{\lambda_n}$.
We can use the Euler formulas to express the these expressions
in terms of $sin$ and $cos$.
{\color{red} Is the statement correct?}
This is represented as ${\bf F}$, the {\bf fundamental matrix} where column $n$ is
${\bf w}_n e^{\lambda_n}$.
\begin{equation}
{\bf F} (t) = \begin{pmatrix}- \frac{k_{2} \cos{\left(t \theta \right)}}{k_{2} + k_{d}} 
+ \frac{\theta \sin{\left(t \theta \right)}}{k_{2} + k_{d}} & - \frac{k_{2} \sin{\left(t \theta \right)}}{k_{2} + k_{d}} - \frac{\theta \cos{\left(t \theta \right)}}{k_{2} + k_{d}}\\\cos{\left(t \theta \right)} & \sin{\left(t \theta \right)}
\end{pmatrix}
\end{equation}

%%%%%%%%%%%%%%%%%%%%%%%
\section{Solving the Initial Value Problem}
We proceed in the usual way to construct a solution:
\begin{enumerate}
\item Find the
solution to the homogeneous system
\(\dot{\bf x}^H (t) = {\bf A} {\bf x}^H (t)\) using the eigenvectors.
\item
Find a particular solution such that
\(\dot{x}^P (t) = {\bf A} {\bf x}^P (t)\)
\item
\({\bf x} (t) = {\bf x}^H (t) + {\bf x}^P (t)\)
\end{enumerate}

\({\bf x}^H (t) = {\bf F} {\bf c} (t),\) where \({\bf c}\) is a vector
of unknown constants that are determined based on initial conditions.

We assume that \({\bf x}^P (t) = {\bf F}(t) {\bf v}\). This means that
\begin{eqnarray}
\dot{\bf x}^P (t) &= & \dot{\bf F} (t) {\bf v} + {\bf F} (t) \dot{\bf v} \\
\dot{\bf F} (t) {\bf v} + {\bf F} (t) \dot{\bf v} & = & {\bf A} {\bf F}(t) {\bf v} + {\bf u} \\
{\bf A} {\bf F} (t){\bf v} + {\bf F} (t) \dot{\bf v} & = & {\bf A} {\bf F} (t) {\bf v} + {\bf u} \\
{\bf F} (t) \dot{\bf v} & = & {\bf u} \\
{\bf v} = \int \left( {\bf F}^{-1} (t) {\bf u} \right)dt
\end{eqnarray}

    Solving, we have

\begin{eqnarray}
{\bf x}^P (t) & = & {\bf F} (t) {\bf v} \\
& = & \begin{pmatrix}\frac{- k_{2}^{2} k_{4} \cos{\left(t \theta \right)} - k_{2}^{2} k_{4} + k_{2}^{2} k_{6} \cos{\left(t \theta \right)} + k_{2}^{2} k_{6} - k_{2} k_{4} k_{d} \cos{\left(t \theta \right)} - k_{2} k_{4} k_{d} + k_{2} k_{4} \theta \sin{\left(t \theta \right)} - k_{2} k_{6} \theta \sin{\left(t \theta \right)} + k_{4} k_{d} \theta \sin{\left(t \theta \right)} + k_{6} \theta^{2}}{\theta^{2} \left(k_{2} + k_{d}\right)}\\\frac{k_{2} k_{4} \cos{\left(t \theta \right)} + k_{2} k_{4} - k_{2} k_{6} \cos{\left(t \theta \right)} - k_{2} k_{6} + k_{4} k_{d} \cos{\left(t \theta \right)} + k_{4} k_{d}}{\theta^{2}}\end{pmatrix}
\end{eqnarray}

    \begin{eqnarray}
{\bf x} (t) & = & {\bf x}^H (t) + {\bf x}^P (t) \\
& = & \begin{pmatrix}- \frac{k_{2} \cos{\left(t \theta \right)}}{k_{2} + k_{d}} + \frac{\theta \sin{\left(t \theta \right)}}{k_{2} + k_{d}} & - \frac{k_{2} \sin{\left(t \theta \right)}}{k_{2} + k_{d}} - \frac{\theta \cos{\left(t \theta \right)}}{k_{2} + k_{d}}\\\cos{\left(t \theta \right)} & \sin{\left(t \theta \right)}\end{pmatrix}  \begin{pmatrix} c_1 \\ c_2 \end{pmatrix} \\
&  & + \begin{pmatrix}\frac{- k_{2}^{2} k_{4} \cos{\left(t \theta \right)} - k_{2}^{2} k_{4} + k_{2}^{2} k_{6} \cos{\left(t \theta \right)} + k_{2}^{2} k_{6} - k_{2} k_{4} k_{d} \cos{\left(t \theta \right)} - k_{2} k_{4} k_{d} + k_{2} k_{4} \theta \sin{\left(t \theta \right)} - k_{2} k_{6} \theta \sin{\left(t \theta \right)} + k_{4} k_{d} \theta \sin{\left(t \theta \right)} + k_{6} \theta^{2}}{\theta^{2} \left(k_{2} + k_{d}\right)}\\\frac{k_{2} k_{4} \cos{\left(t \theta \right)} + k_{2} k_{4} - k_{2} k_{6} \cos{\left(t \theta \right)} - k_{2} k_{6} + k_{4} k_{d} \cos{\left(t \theta \right)} + k_{4} k_{d}}{\theta^{2}}\end{pmatrix}
\end{eqnarray}

    We find \(c_1, c_2\) by \begin{eqnarray}
{\bf x} (0) & = & \begin{pmatrix} x_1 (0) \\ x_2 (0) \end{pmatrix} \\
& = & \begin{pmatrix}- \frac{k_{2} \cos{\left(t \theta \right)}}{k_{2} + k_{d}} + \frac{\theta \sin{\left(t \theta \right)}}{k_{2} + k_{d}} & - \frac{k_{2} \sin{\left(t \theta \right)}}{k_{2} + k_{d}} - \frac{\theta \cos{\left(t \theta \right)}}{k_{2} + k_{d}}\\\cos{\left(t \theta \right)} & \sin{\left(t \theta \right)}\end{pmatrix}  \begin{pmatrix} c_1 \\ c_2 \end{pmatrix} \\
&  & + \begin{pmatrix}\frac{- k_{2}^{2} k_{4} \cos{\left(t \theta \right)} - k_{2}^{2} k_{4} + k_{2}^{2} k_{6} \cos{\left(t \theta \right)} + k_{2}^{2} k_{6} - k_{2} k_{4} k_{d} \cos{\left(t \theta \right)} - k_{2} k_{4} k_{d} + k_{2} k_{4} \theta \sin{\left(t \theta \right)} - k_{2} k_{6} \theta \sin{\left(t \theta \right)} + k_{4} k_{d} \theta \sin{\left(t \theta \right)} + k_{6} \theta^{2}}{\theta^{2} \left(k_{2} + k_{d}\right)}\\\frac{k_{2} k_{4} \cos{\left(t \theta \right)} + k_{2} k_{4} - k_{2} k_{6} \cos{\left(t \theta \right)} - k_{2} k_{6} + k_{4} k_{d} \cos{\left(t \theta \right)} + k_{4} k_{d}}{\theta^{2}}\end{pmatrix}
\end{eqnarray}

Solving, we have
\({\bf x} (t) = \begin{pmatrix}\frac{\left(- \frac{k_{2} \sin{\left(t \theta \right)}}{k_{2} + k_{d}} - \frac{\theta \cos{\left(t \theta \right)}}{k_{2} + k_{d}}\right) \left(- k_{2} x_{1 0} - k_{2} x_{2 0} + k_{6} - k_{d} x_{1 0}\right)}{\theta} + \frac{\left(- \frac{k_{2} \cos{\left(t \theta \right)}}{k_{2} + k_{d}} + \frac{\theta \sin{\left(t \theta \right)}}{k_{2} + k_{d}}\right) \left(- 2 k_{2} k_{4} + 2 k_{2} k_{6} - 2 k_{4} k_{d} + \theta^{2} x_{2 0}\right)}{\theta^{2}} + \frac{- k_{2}^{2} k_{4} \cos{\left(t \theta \right)} - k_{2}^{2} k_{4} + k_{2}^{2} k_{6} \cos{\left(t \theta \right)} + k_{2}^{2} k_{6} - k_{2} k_{4} k_{d} \cos{\left(t \theta \right)} - k_{2} k_{4} k_{d} + k_{2} k_{4} \theta \sin{\left(t \theta \right)} - k_{2} k_{6} \theta \sin{\left(t \theta \right)} + k_{4} k_{d} \theta \sin{\left(t \theta \right)} + k_{6} \theta^{2}}{\theta^{2} \left(k_{2} + k_{d}\right)}\\ \frac{\left(- k_{2} x_{1 0} - k_{2} x_{2 0} + k_{6} - k_{d} x_{1 0}\right) \sin{\left(t \theta \right)}}{\theta} + \frac{\left(- 2 k_{2} k_{4} + 2 k_{2} k_{6} - 2 k_{4} k_{d} + \theta^{2} x_{2 0}\right) \cos{\left(t \theta \right)}}{\theta^{2}} + \frac{k_{2} k_{4} \cos{\left(t \theta \right)} + k_{2} k_{4} - k_{2} k_{6} \cos{\left(t \theta \right)} - k_{2} k_{6} + k_{4} k_{d} \cos{\left(t \theta \right)} + k_{4} k_{d}}{\theta^{2}}\end{pmatrix}\)

    Our next task is to restructure \({\bf x} (t)\) to isolate the
oscillation characteristics \(\theta, \alpha_n, \phi_n, \omega_n\). This
is mostly a tedious factoring. \(\theta\) is the coefficient of time
\(t\). The \(\alpha_n\) are obtained from the coefficients of
\(cos(\theta t)\) and \(sin(\theta t)\) in \(x_n (t)\). And,
\(\omega_n\) are terms in \(x_n (t)\) that have no \(sin\) or \(cos\).
We obtain \(\phi_n\) by applying the trigonometric equality
\[a cos(t) + b sin(t) = \sqrt{a^2 + b^2} sin(t + tan^{-1}\frac{a}{b})
\].

    The results are:

\begin{itemize}
\item
  \(\theta = \sqrt{k_2 k_d}\)
\item
  \(\alpha_1 = \frac{\sqrt{\theta^{2} \left(k_{2}^{2} x_{1 0} + k_{2}^{2} x_{2 0} - k_{2} k_{4} + k_{2} k_{d} x_{1 0} - k_{4} k_{d} + \theta^{2} x_{2 0}\right)^{2} + \left(k_{2}^{2} k_{4} - k_{2}^{2} k_{6} + k_{2} k_{4} k_{d} + k_{2} \theta^{2} x_{1 0} - k_{6} \theta^{2} + k_{d} \theta^{2} x_{1 0}\right)^{2}}}{\theta^{2} \left(k_{2} + k_{d}\right)}\)
\item
  \(\alpha_2 = \frac{\sqrt{\theta^{2} \left(k_{2} x_{1 0} + k_{2} x_{2 0} - k_{6} + k_{d} x_{1 0}\right)^{2} + \left(k_{2} k_{4} - k_{2} k_{6} + k_{4} k_{d} - \theta^{2} x_{2 0}\right)^{2}}}{\theta^{2}}\)
\item
  \$\phi\_1 =
  \operatorname{atan}{\left(\frac{k_{2}^{2} k_{4} - k_{2}^{2} k_{6} + k_{2} k_{4} k_{d} + k_{2} \theta^{2} x_{1 0} - k_{6} \theta^{2} + k_{d} \theta^{2} x_{1 0}}{\theta \left(k_{2}^{2} x_{1 0} + k_{2}^{2} x_{2 0} - k_{2} k_{4} + k_{2} k_{d} x_{1 0} - k_{4} k_{d} + \theta^{2} x_{2 0}\right)} \right)}
  + \delta\_1 \pi,\textasciitilde{} \$ where
  \textbackslash begin\{eqnarray\} \delta\_1 \& = \&
  \frac{k_{2}^{2} x_{1 0}}{k_{2} \theta + k_{d} \theta} +
  \frac{k_{2}^{2} x_{2 0}}{k_{2} \theta + k_{d} \theta} +
  \frac{k_{2} k_{4} \theta}{k_{2} \theta^{2} + k_{d} \theta^{2}}
\item
  \frac{2 k_{2} k_{4}}{k_{2} \theta + k_{d} \theta} -
  \textbackslash frac\{k\_\{2\} k\_\{6\} \theta\}\{k\_\{2\}
  \theta\^{}\{2\}
\item
  k\_\{d\} \theta\^{}\{2\}\} \textbackslash{} \& \&
\item
  \frac{k_{2} k_{6}}{k_{2} \theta + k_{d} \theta} +
  \frac{k_{2} k_{d} x_{1 0}}{k_{2} \theta + k_{d} \theta} +
  \frac{k_{4} k_{d} \theta}{k_{2} \theta^{2} + k_{d} 
  \theta^{2}} - \frac{2 k_{4} k_{d}}{k_{2} \theta + k_{d} \theta} +
  \frac{\theta x_{2 0}}{k_{2} + k_{d}} \textless{} 0
  \textbackslash end\{eqnarray\}
\item
  \$\phi\_2 =
  \operatorname{atan}{\left(\frac{k_{2} k_{4} - k_{2} k_{6} + k_{4} k_{d} - \theta^{2} x_{2 0}}{\theta \left(k_{2} x_{1 0} + k_{2} x_{2 0} - k_{6} + k_{d} x_{1 0}\right)} \right)}
  + \delta\_2 \pi,\textasciitilde{} \$ where
  \(\delta_2 = \frac{k_{2} x_{1 0}}{\theta} + \frac{k_{2} x_{2 0}}{\theta} - \frac{k_{6}}{\theta} + \frac{k_{d} x_{1 0}}{\theta} > 0\)
\item
  \(\omega_1 = \frac{- k_{2}^{2} k_{4} + k_{2}^{2} k_{6} - k_{2} k_{4} k_{d} + k_{6} \theta^{2}}{k_{2} \theta^{2} + k_{d} \theta^{2}}\)
\item
  \(\omega_2 = \frac{k_{2} k_{4} - k_{2} k_{6} + k_{4} k_{d}}{\theta^{2}}\)
\end{itemize}


    % Add a bibliography block to the postdoc
    
    
    
\end{document}
