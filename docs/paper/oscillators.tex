%% BioMed_Central_Tex_Template_v1.06
%%                                      %
%  bmc_article.tex            ver: 1.06 %
%                                       %

%%IMPORTANT: do not delete the first line of this template
%%It must be present to enable the BMC Submission system to
%%recognise this template!!

%%%%%%%%%%%%%%%%%%%%%%%%%%%%%%%%%%%%%%%%%
%%                                     %%
%%  LaTeX template for BioMed Central  %%
%%     journal article submissions     %%
%%                                     %%
%%          <8 June 2012>              %%
%%                                     %%
%%                                     %%
%%%%%%%%%%%%%%%%%%%%%%%%%%%%%%%%%%%%%%%%%

%%%%%%%%%%%%%%%%%%%%%%%%%%%%%%%%%%%%%%%%%%%%%%%%%%%%%%%%%%%%%%%%%%%%%
%%                                                                 %%
%% For instructions on how to fill out this Tex template           %%
%% document please refer to Readme.html and the instructions for   %%
%% authors page on the biomed central website                      %%
%% https://www.biomedcentral.com/getpublished                      %%
%%                                                                 %%
%% Please do not use \input{...} to include other tex files.       %%
%% Submit your LaTeX manuscript as one .tex document.              %%
%%                                                                 %%
%% All additional figures and files should be attached             %%
%% separately and not embedded in the \TeX\ document itself.       %%
%%                                                                 %%
%% BioMed Central currently use the MikTex distribution of         %%
%% TeX for Windows) of TeX and LaTeX.  This is available from      %%
%% https://miktex.org/                                             %%
%%                                                                 %%
%%%%%%%%%%%%%%%%%%%%%%%%%%%%%%%%%%%%%%%%%%%%%%%%%%%%%%%%%%%%%%%%%%%%%

%%% additional documentclass options:
%  [doublespacing]
%  [linenumbers]   - put the line numbers on margins

%%% loading packages, author definitions

%\documentclass[twocolumn]{bmcart}% uncomment this for twocolumn layout and comment line below
\documentclass{bmcart}

%%% Load packages
\usepackage{amsthm,amsmath}
%\RequirePackage[numbers]{natbib}
%\RequirePackage[authoryear]{natbib}% uncomment this for author-year bibliography
%\RequirePackage{hyperref}
\usepackage[utf8]{inputenc} %unicode support
%\usepackage[applemac]{inputenc} %applemac support if unicode package fails
%\usepackage[latin1]{inputenc} %UNIX support if unicode package fails

%for abbreviations item list
\usepackage{blindtext}
\usepackage{enumitem}
\usepackage{xcolor}
%for url
%\usepackage{hyperref}
%
%include figure 
\usepackage{graphicx}

%%%%%%%%%%%%%%%%%%%%%%%%%%%%%%%%%%%%%%%%%%%%%%%%%
%%                                             %%
%%  If you wish to display your graphics for   %%
%%  your own use using includegraphic or       %%
%%  includegraphics, then comment out the      %%
%%  following two lines of code.               %%
%%  NB: These line *must* be included when     %%
%%  submitting to BMC.                         %%
%%  All figure files must be submitted as      %%
%%  separate graphics through the BMC          %%
%%  submission process, not included in the    %%
%%  submitted article.                         %%
%%                                             %%
%%%%%%%%%%%%%%%%%%%%%%%%%%%%%%%%%%%%%%%%%%%%%%%%%

%\def\includegraphic{}
%\def\includegraphics{}

%%% Put your definitions there:
\startlocaldefs
\endlocaldefs

\usepackage{xcolor}
\newcommand{\comment}[1]{ \color{red} #1 \color{black}}
\newcommand{\reply}[1]{ \color{blue} [#1] \color{black}}

%%% Begin ...
\begin{document}

%%% Start of article front matter
\begin{frontmatter}

\begin{fmbox}

\dochead{Research}

%%%%%%%%%%%%%%%%%%%%%%%%%%%%%%%%%%%%%%%%%%%%%%
%%                                          %%
%% Enter the title of your article here     %%
%%                                          %%
%%%%%%%%%%%%%%%%%%%%%%%%%%%%%%%%%%%%%%%%%%%%%%

\title{A Mathematical Framework for Building Oscillators in Reaction Networks}

%%%%%%%%%%%%%%%%%%%%%%%%%%%%%%%%%%%%%%%%%%%%%%
%%                                          %%
%% Enter the authors here                   %%
%%                                          %%
%% Specify information, if available,       %%
%% in the form:                             %%
%%   <key>={<id1>,<id2>}                    %%
%%   <key>=                                 %%
%% Comment or delete the keys which are     %%
%% not used. Repeat \author command as much %%
%% as required.                             %%
%%                                          %%
%%%%%%%%%%%%%%%%%%%%%%%%%%%%%%%%%%%%%%%%%%%%%%

\author[
  addressref={aff2},                   % id's of address
  %{aff1,aff2}
]{\inits{J.H.}\fnm{Hellerstein} \snm{Joseph}}

%%%%%%%%%%%%%%%%%%%%%%%%%%%%%%%%%%%%%%%%%%%%%%
%%                                          %%
%% Enter the authors' addresses here        %%
%%                                          %%
%% Repeat \address commands as much as      %%
%% required.                                %%
%%                                          %%
%%%%%%%%%%%%%%%%%%%%%%%%%%%%%%%%%%%%%%%%%%%%%%

\address[id=aff2]{%
  \orgdiv{eScience Institute},
  \orgname{University of Washington},
  %\street{},
  %\postcode{}
  \city{Seattle},
  \cny{USA}
}

%%%%%%%%%%%%%%%%%%%%%%%%%%%%%%%%%%%%%%%%%%%%%%
%%                                          %%
%% Enter short notes here                   %%
%%                                          %%
%% Short notes will be after addresses      %%
%% on first page.                           %%
%%                                          %%
%%%%%%%%%%%%%%%%%%%%%%%%%%%%%%%%%%%%%%%%%%%%%%

%\begin{artnotes}
%%\note{Sample of title note}     % note to the article
%\note[id=n1]{Equal contributor} % note, connected to author
%\end{artnotes}

\end{fmbox}% comment this for two column layout

%%%%%%%%%%%%%%%%%%%%%%%%%%%%%%%%%%%%%%%%%%%%%%%
%%                                           %%
%% The Abstract begins here                  %%
%%                                           %%
%% Please refer to the Instructions for      %%
%% authors on https://www.biomedcentral.com/ %%
%% and include the section headings          %%
%% accordingly for your article type.        %%
%%                                           %%
%%%%%%%%%%%%%%%%%%%%%%%%%%%%%%%%%%%%%%%%%%%%%%%

\begin{abstractbox}

\begin{abstract} % abstract


\end{abstract}

%%%%%%%%%%%%%%%%%%%%%%%%%%%%%%%%%%%%%%%%%%%%%%
%%                                          %%
%% The keywords begin here                  %%
%%                                          %%
%% Put each keyword in separate \kwd{}.     %%
%%                                          %%
%%%%%%%%%%%%%%%%%%%%%%%%%%%%%%%%%%%%%%%%%%%%%%

\begin{keyword}
\kwd{Systems biology}
\end{keyword}

% MSC classifications codes, if any
%\begin{keyword}[class=AMS]
%\kwd[Primary ]{}
%\kwd{}
%\kwd[; secondary ]{}
%\end{keyword}

\end{abstractbox}
%
%\end{fmbox}% uncomment this for two column layout

\end{frontmatter}

%%%%%%%%%%%%%%%%%%%%%%%%%%%%%%%%%%%%%%%%%%%%%%%%
%%                                            %%
%% The Main Body begins here                  %%
%%                                            %%
%% Please refer to the instructions for       %%
%% authors on:                                %%
%% https://www.biomedcentral.com/getpublished %%
%% and include the section headings           %%
%% accordingly for your article type.         %%
%%                                            %%
%% See the Results and Discussion section     %%
%% for details on how to create sub-sections  %%
%%                                            %%
%% use \cite{...} to cite references          %%
%%  \cite{koon} and                           %%
%%  \cite{oreg,khar,zvai,xjon,schn,pond}      %%
%%                                            %%
%%%%%%%%%%%%%%%%%%%%%%%%%%%%%%%%%%%%%%%%%%%%%%%%

%%%%%%%%%%%%%%%%%%%%%%%%% start of article main body
% <put your article body there>

%%%%%%%%%%%%%%%%
%% Background %%
%%


\section*{Background}
\begin{enumerate}
\item Importance of oscillators in biology: circadian rhythms, filters, ... Indicate characteristics of importance: frequency, amplitude, phase. Controlling separately is tunable.
\item Models of biological oscillators. Number of species. Rate laws.
   \begin{enumerate}
       \item validation
       \item requirements of oscillation (sufficiently non-linear)
       \item insights into tunable
   \end{enumerate}
\item Requirements of theoretical oscillator: biological credibility
\begin{enumerate}
    \item Non-negative concentrations
    \item Credible kinetics
\end{enumerate}
\item Summary of contribution
\begin{enumerate}
    \item Oscillatory reaction network with linear rate laws. ODEs are a system of linear differential equations. Counter example to claim that the reaction network must be "sufficiently non-linear".
    \item Closed form solution for the time domain behavior of the 2 species linear oscillatory network. Since linear network, initial conditions matter. There are no limit cycles.
    \item The closed form solution provides a mathematical framework that provides insights into tuning non-linear oscillators. Changing amplitude while keeping frequency constant. Phase shifts.
\end{enumerate}

\end{enumerate}


\section*{Methods}
\begin{enumerate}
    \item Jacobian for 2 species, two reactions, mass action
    \item Requirements for sustained oscillation: $T = 0$, $D > 0$
    \item $T> 0$: self catalyzing
    \item $D > 0$: inhibition through degradation. Requires care on operating region.
    \item Solution to homogeneous system. Issue -- need an offset. Calculating solution vectors using imaginary and real parts of eigenvectors.
    \item Particular solution.
    \item Full solution. Requires trig equality.
\end{enumerate}


\section*{Results}
\begin{enumerate}
    \item Validation of the solution via simulations
    \item What parts of the oscillator can be removed if unconcerned about certain elements of control like amplitude and phase?
    \item Interpretations
    \begin{enumerate}
        \item Must have $u_1, u_2$ be non-zero in order to get an offset so that there are non-zero values for the two species.
        \item Large frequency ($\alpha$) approximation
        \item Large $\delta$ approximation
    \end{enumerate}
    \item Robustness
\end{enumerate}


\section*{Discussion}


\section*{Conclusions}


%\section*{Appendix}
%Text for this section\ldots

%%%%%%%%%%%%%%%%%%%%%%%%%%%%%%%%%%%%%%%%%%%%%%
%%                                          %%
%% Backmatter begins here                   %%
%%                                          %%
%%%%%%%%%%%%%%%%%%%%%%%%%%%%%%%%%%%%%%%%%%%%%%

\begin{backmatter}

\section*{Acknowledgements}%% if any
We thank Lucian Smith for the discussion about the kinetic law formula expansion based on libSBML. We also appreciate Herbert M. Sauro’s discussions regarding kinetics and synthetic reaction networks cited as the references.

\section*{Funding}%% if any
Research reported in this was supported by NIBIB of the National Institutes of Health under award number U24EB028887.

\section*{Abbreviations}%% if any
Our two dimensional kinetics classification scheme (2DK) organizes reactions along two dimensions: kinetics type (K type) and reaction type (R type).\\
I. Kinetics type (K type)\\
K type including ten types:
\begin{description}[font=$\bullet$~\normalfont\scshape\color{red!50!black}]
\item ZERO: Zeroth order;
\item UNDR: Uni-directional mass action;
\item UNMO: Uni-term with moderator;
\item BIDR: Bi-directional mass action;
\item BIMO: Bi-terms with moderator;
\item MM: Michaelis-Menten kinetics without an explicit enzyme;
\item MMCAT: Michaelis-Menten kinetics with an explicit enzyme;
\item HILL: Hill kinetics;
\item FR: Kinetic law in the fraction format other than MM, MMCAT or HILL;
\item NA: not classified.
\end{description}
II. Reaction type (R type)\\
R type is quantitatively represented by the number of reactants (R = 0, 1, 2, $>$2) and products (P= 0, 1, 2, $>$2)).

\section*{Availability and Implementation}%% if any
{\texttt SBMLKinetics} is a publicly available Python package (https://pypi.org/project/SBMLKinetics/) and licensed under the liberal MIT open-source license. The package is available for all major platforms. The source code has been deposited at GitHub (https://github.com/ModelEngineering/kinetics\_validator). Users can install the package using the standard pip installation mechanism: pip install SBMLKinetics. The package is fully documented as https://modelengineering.github.io/kinetics\_validator/. 

\section*{Availability of data and materials}%% if any
The data sets generated and/or analyzed during the current study are available in the Github repository, https://github.com/ModelEngineering/kinetics\_validator/tree/master/data.

\section*{Competing interests}
The authors declare that they have no competing interests.

%\section*{Consent for publication}%% if any
%Text for this section\ldots

\section*{Authors' contributions}
JX implemented the project, did the data analysis and was the main developer of the Python package. JH was responsible for conception and project administration. JX and JH wrote the manuscript. All authors read and approved the final manuscript.

%\section*{Authors' information}%% if any
%Text for this section\ldots

%%%%%%%%%%%%%%%%%%%%%%%%%%%%%%%%%%%%%%%%%%%%%%%%%%%%%%%%%%%%%
%%                  The Bibliography                       %%
%%                                                         %%
%%  Bmc_mathpys.bst  will be used to                       %%
%%  create a .BBL file for submission.                     %%
%%  After submission of the .TEX file,                     %%
%%  you will be prompted to submit your .BBL file.         %%
%%                                                         %%
%%                                                         %%
%%  Note that the displayed Bibliography will not          %%
%%  necessarily be rendered by Latex exactly as specified  %%
%%  in the online Instructions for Authors.                %%
%%                                                         %%
%%%%%%%%%%%%%%%%%%%%%%%%%%%%%%%%%%%%%%%%%%%%%%%%%%%%%%%%%%%%%

% if your bibliography is in bibtex format, use those commands:
\bibliographystyle{bmc-mathphys} % Style BST file (bmc-mathphys, vancouver, spbasic).
\bibliography{bmc_article}      % Bibliography file (usually '*.bib' )
% for author-year bibliography (bmc-mathphys or spbasic)
% a) write to bib file (bmc-mathphys only)
% @settings{label, options="nameyear"}
% b) uncomment next line
%\nocite{label}

% or include bibliography directly:
% \begin{thebibliography}
% \bibitem{b1}
% \end{thebibliography}

%%%%%%%%%%%%%%%%%%%%%%%%%%%%%%%%%%%
%%                               %%
%% Figures                       %%
%%                               %%
%% NB: this is for captions and  %%
%% Titles. All graphics must be  %%
%% submitted separately and NOT  %%
%% included in the Tex document  %%
%%                               %%
%%%%%%%%%%%%%%%%%%%%%%%%%%%%%%%%%%%

%%
%% Do not use \listoffigures as most will included as separate files

%\section*{Figures}
%\begin{figure}[h!]
%  \caption{Sample figure title}
%\end{figure}

%\begin{figure}[h!]
%  \caption{Sample figure title}
%\end{figure}

%%%%%%%%%%%%%%%%%%%%%%%%%%%%%%%%%%%
%%                               %%
%% Tables                        %%
%%                               %%
%%%%%%%%%%%%%%%%%%%%%%%%%%%%%%%%%%%

%% Use of \listoftables is discouraged.
%%
%\section*{Tables}
%\begin{table}[h!]
%\caption{Sample table title. This is where the description of the table %should go}
%  \begin{tabular}{cccc}
%    \hline
%    & B1  &B2   & B3\\ \hline
%    A1 & 0.1 & 0.2 & 0.3\\
%    A2 & ... & ..  & .\\
%    A3 & ..  & .   & .\\ \hline
%  \end{tabular}
%\end{table}

%%%%%%%%%%%%%%%%%%%%%%%%%%%%%%%%%%%
%%                               %%
%% Additional Files              %%
%%                               %%
%%%%%%%%%%%%%%%%%%%%%%%%%%%%%%%%%%%

%\section*{Additional Files}
%  \subsection*{Additional file 1 --- Sample additional file title}
%    Additional file descriptions text (including details of how to
%    view the file, if it is in a non-standard format or the file extension). % This might
%    refer to a multi-page table or a figure.
%
%  \subsection*{Additional file 2 --- Sample additional file title}
%    Additional file descriptions text.

\end{backmatter}
\end{document}
