\NeedsTeXFormat{LaTeX2e}[1995/12/01]
\documentclass[11pt]{elsarticle}

% Load packages
\usepackage{listings}
\usepackage{natbib}
\usepackage{url}  % Formatting web addresses
\usepackage{multicol}   %Columns
\usepackage{subcaption}
%\usepackage[utf8]{inputenc} %unicode support
\usepackage{graphicx}
\usepackage{xcolor}
\usepackage{tikz}
\usetikzlibrary{arrows,arrows.meta,plotmarks}
\usetikzlibrary{matrix}
\usepgflibrary{shapes}
\usetikzlibrary{decorations.pathmorphing}
\usetikzlibrary{calc}
\urlstyle{rm}
\usepackage{courier}
\usepackage{soul}

\definecolor{mygreen}{rgb}{0,0.6,0}
\definecolor{mygray}{rgb}{0.5,0.5,0.5}
\definecolor{mymauve}{rgb}{0.58,0,0.82}
\definecolor{anti-flashwhite}{rgb}{0.95, 0.95, 0.96}
\definecolor{aliceblue}{rgb}{0.94, 0.97, 1.0}
\definecolor{sg-comment}{RGB}{255,127,80}

\lstset{
         basicstyle=\footnotesize\ttfamily,
         numberstyle=\tiny,
         numbersep=5pt,
         tabsize=2,
         extendedchars=true,
         breaklines=true,
         keywordstyle=\color{red},
         stringstyle=\color{white}\ttfamily,
         showspaces=false,
         showtabs=false,
         xleftmargin=17pt,
         framexleftmargin=17pt,
         framexrightmargin=5pt,
         framexbottommargin=4pt,
         stringstyle=\color{mymauve},
         showstringspaces=false
 }

% \lstset{ %
%   backgroundcolor=\color{aliceblue},   % choose the background color
%   basicstyle=\ttfamily\small, % size of fonts used for the code
%   breaklines=true,                 % automatic line breaking only at whitespace
%   captionpos=b,                    % sets the caption-position to bottom
%   commentstyle=\color{mygreen},    % comment style
%   escapeinside={\%*}{*)},          % if you want to add LaTeX within your code
%   keywordstyle=\color{blue},       % keyword style
%   stringstyle=\color{mymauve},     % string literal style
%   showstringspaces=false
% }

\newcommand{\bN}{\mathbf N}
\newcommand{\bJ}{\mathbf J}
\newcommand{\bv}{\mathbf v}
\newcommand{\bx}{\mathbf x}
\newcommand{\bk}{\mathbf k}
\newcommand{\bLo}{\mathbf L_0}
\newcommand{\bNo}{\mathbf N_0}
\newcommand{\bNr}{\mathbf N_R}
\newcommand{\bNIC}{\mathbf N_{IC}}
\newcommand{\bNDC}{\mathbf N_{DC}}
\newcommand{\bJM}{\mathbf J_M}
\newcommand{\bJC}{\mathbf J_C}
\newcommand{\bK}{\mathbf K}
\newcommand{\bKo}{\mathbf K_0}
\newcommand{\bI}{\mathbf I}
\newcommand{\bZero}{\mathbf 0}
\newcommand{\bL}{\mathbf L}
\newcommand{\bGamma}{\mathbf \Gamma}
\newcommand{\bT}{\mathbf T}
\newcommand{\bQ}{\mathbf Q}
\newcommand{\bR}{\mathbf R}
\newcommand{\bU}{\mathbf U}
\newcommand{\bve}{\textbf{e}}

\setlength{\parskip}{1em}
\setlength{\parindent}{0em}

% Begin ...
\begin{document}

\title{A Mathematical Framework for Building Oscillators in Reaction Networks}

\author[UWBIOE]{Joseph L. Hellerstein}
\ead{jlheller@uw.edu}


\address[UWBIOE]{eScience Institute, University of Washington, Allen School of Computer Science, Seattle, WA, USA, 98195}

\begin{abstract} 
Abstract
\end{abstract}

\begin{keyword}
Simulation \sep SBML \sep Software \sep Systems Biology
\end{keyword}

\maketitle

\section*{Introduction}
\begin{enumerate}
\item Importance of oscillators in biology: circadian rhythms, filters, ... Indicate characteristics of importance: frequency, amplitude, phase. Controlling separately is tunable.
\item Models of biological oscillators. Number of species. Rate laws.
   \begin{enumerate}
       \item validation
       \item requirements of oscillation (sufficiently non-linear)
       \item insights into tunable
   \end{enumerate}
\item Requirements of theoretical oscillator: biological credibility
\begin{enumerate}
    \item Non-negative concentrations
    \item Credible kinetics
\end{enumerate}
\item Summary of contribution
\begin{enumerate}
    \item Oscillatory reaction network with linear rate laws. ODEs are a system of linear differential equations. Counter example to claim that the reaction network must be "sufficiently non-linear".
    \item Closed form solution for the time domain behavior of the 2 species linear oscillatory network. Since linear network, initial conditions matter. There are no limit cycles.
    \item The closed form solution provides a mathematical framework that provides insights into tuning non-linear oscillators. Changing amplitude while keeping frequency constant. Phase shifts.
\end{enumerate}

\end{enumerate}

\section*{Methods}
\begin{enumerate}
    \item Jacobian for 2 species, two reactions, mass action
    \item Requirements for sustained oscillation: $T = 0$, $D > 0$
    \item $T> 0$: self catalyzing
    \item $D > 0$: inhibition through degradation. Requires care on operating region.
    \item Solution to homogeneous system. Issue -- need an offset. Calculating solution vectors using imaginary and real parts of eigenvectors.
    \item Particular solution.
    \item Full solution. Requires trig equality.
\end{enumerate}

\section*{Results}
\begin{enumerate}
    \item Validation of the solution via simulations
    \item What parts of the oscillator can be removed if unconcerned about certain elements of control like amplitude and phase?
    \item Interpretations
    \begin{enumerate}
        \item Must have $u_1, u_2$ be non-zero in order to get an offset so that there are non-zero values for the two species.
        \item Large frequency ($\alpha$) approximation
        \item Large $\delta$ approximation
    \end{enumerate}
    \item Robustness
\end{enumerate}

\section*{Conclusions}


\section*{References}

\bibliographystyle{elsarticle-harv.bst}  % Style BST file
\bibliography{references}     % Bibliography file (usually '*.bib' )

\end{document}
